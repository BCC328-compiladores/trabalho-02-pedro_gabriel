\documentclass[12pt,a4paper]{article}
\usepackage[utf8]{inputenc}
\usepackage[brazil]{babel}
\usepackage[T1]{fontenc}
\usepackage{geometry}
\usepackage{graphicx}
\usepackage{hyperref}
\usepackage{listings}
\usepackage{xcolor}
\usepackage{amsmath}
\usepackage{amssymb}

\geometry{a4paper, left=3cm, right=2cm, top=3cm, bottom=2cm}

\definecolor{codegreen}{rgb}{0,0.6,0}
\definecolor{codegray}{rgb}{0.5,0.5,0.5}
\definecolor{codepurple}{rgb}{0.58,0,0.82}
\definecolor{backcolour}{rgb}{0.95,0.95,0.92}

\lstdefinestyle{mystyle}{
    backgroundcolor=\color{backcolour},
    commentstyle=\color{codegreen},
    keywordstyle=\color{magenta},
    numberstyle=\tiny\color{codegray},
    stringstyle=\color{codepurple},
    basicstyle=\ttfamily\footnotesize,
    breakatwhitespace=false,
    breaklines=true,
    captionpos=b,
    keepspaces=true,
    numbers=left,
    numbersep=5pt,
    showspaces=false,
    showstringspaces=false,
    showtabs=false,
    tabsize=2
}

\lstset{style=mystyle}

\title{Relatório de Projeto: Compilador SL}
\author{
  Pedro Augusto Sousa Gonçalves \\
  \small{Matrícula: 21.1.4015}
  \and
  Gabriel Carlos Silva \\
  \small{Matrícula: 23.1.4016}
}

\date{
    BCC328 - Construção de Compiladores I \\
    DECOM - UFOP \\
    \today
}

\begin{document}

\maketitle

\begin{abstract}
Este trabalho apresenta o desenvolvimento das etapas de análise léxica e sintática de um compilador para a linguagem \textit{SL}. O projeto foi desenvolvido como requisito parcial da disciplina de Construção de Compiladores I. A implementação utilizou a linguagem de programação \texttt{Haskell} em conjunto com as ferramentas geradoras \texttt{Alex} e \textit{Happy}. O analisador resultante é capaz de identificar tokens válidos e verificar a conformidade do código fonte com a gramática livre de contexto especificada, reportando erros léxicos e sintáticos quando encontrados.
\end{abstract}

\clearpage\newpage
\tableofcontents
\clearpage\newpage

\section{Introdução}
O estudo e o desenvolvimento de compiladores é um dos temas mais importantes da Ciência da Computação, pois envolve a integração de conceitos fundamentais da teoria da computação, como linguagens formais, teoria de autômatos, estrutura de dados, algoritmos e arquitetura de computadores. Um compilador é responsável por traduzir programas escritos em uma linguagem de alto nível para uma representação de mais baixo nível, preservando a semântica do programa original e permitindo sua execução.

Este trabalho tem como objetivo o desenvolvimento de um compilador para a linguagem SL, uma linguagem simplificada, com suporte a tipagem estática, funções, estruturas de controle, arranjos e registros. O compilador será implementado na linguagem Haskell, e terá como linguagem-alvo o WebAssembly (WAT), possibilitando a execução dos programas gerados em navegadores web.

O projeto foi sub-dividido em etapas incrementais: Na primeira etapa, foco deste relatório, são abordadas as fases de análise léxica e análise sintática, responsáveis por transformar o código-fonte em uma estrutura sintática que represente formalmente o programa de entrada. Para isso, foram utilizadas as ferramentas Alex e Happy, amplamente empregadas na costrução de compiladores em Haskell.

Ao longo deste relatório, são descritas as principais decisões de projeto adotadas, a gramática definida para a linguagem SL e a arquitetura geral dos analisadores léxico e sintático. O objetivo é não apenas apresentar a implementação, mas também justificar as escolhas realizadas, relacionando-as com os conceitos teóricos estudados na disciplina de Construção de Compiladores.
\clearpage\newpage



\section{Metodologia}
\subsection{Estrutura sintática de SL}
Definir a estrutura sintática da linguagem é um passo fundamental no desenvolvimento do compilador, pois estabelece formalmente como programas válidos da linguagem devem ser escritos.

A linguagem SL adota uma sintaxe com suporte a declarações explícitas de variáveis, definição de funções, estruturas de controle (como condicionais e laços de repetição), além de arranjos e registros. Cada programa é composto por um conjunto de definições globais, que podem ser declarações de funções ou de estruturas. As funções possuem parâmetros, tipo de retorno e um corpo delimitado por blocos.


\subsection{Sistema de tipos para SL}
A linguagem SL utiliza um sistema de tipos estático, no qual os tipos das expressões e identificadores são verificados em tempo de compilação. Os tipos básicos suportados incluem 'int", "float", "bool" e "string", além de tipos compostos como arranjos registros e funções. Esse sistema tem como objetivo aumentar a segurança do programa, prevenindo erros comuns como operações inválidas entre tipos incompatíveis.

Além da tipagem estática, SL oferece suporte a polimorfismo paramétrico, permitindo a definição de funções genéricas. 



\subsection{Inferência de tipos para SL}
A linguagem SL foi projetada com suporte à inferência de tipos, permitindo que o compilador deduza automaticamente o tipo de expressões e identificadores a partir do contexto em que são utilizados. Esse mecanismo reduz a necessidade de anotações explícitas por parte do programador, tornando o código mais conciso, sem comprometer a segurança.

A inferência de tipos em SL baseia-se na análise das expressões durante a construção e posterior anotação da Árvore de Sintaxe Abstrata (AST). A cada nó da AST é associado um tipo, que pode ser inicialmente desconhecido e progressivamente refinado à medida que regras de tipagem são aplicadas. Por exemplo, em uma expressão aritmética, o tipo do resultado é inferido a partir dos tipos de seus operandos e do operador utilizado.

No caso de variáveis locais, o tipo pode ser inferido diretamente a partir da expressão de inicialização. Já em chamadas de função, o compilador verifica a compatibilidade entre os tipos dos argumentos e os tipos esperados pelos parâmetros da função, propagando o tipo de retorno conforme definido ou inferido. Para funções genéricas, o processo envolve a instância de variáveis de tipo, respeitando as restrições impostas pelo uso da função.

Apesar do suporte à inferência, SL exige anotações explícitas em pontos estruturais relevantes, como definições de funções globais e declarações de estruturas. Como ilustração, considere o código a seguir:

\begin{lstlisting}[numbers=none, language=C]
let x = 10;
let y = x + 2;
\end{lstlisting}

Neste exemplo, o compilador infere o tipo mais apropriado para as variáveis x e y, que no caso é o tipo inteiro.

\subsection{Semântica operacional para SL}
A semântica operacional da linguagem SL descreve formalmente o comportamento dos programas durante sua execução, especificando como cada construção sintática afeta o estado do programa. Esse estado pode ser compreendido, de forma abstrata, como a combinação de um ambiente de variáveis, uma pilha de chamadas de função e uma memória responsável pelo armazenamento de valores, incluindo arranjos e registros.

A execução de comandos em SL ocorre de maneira sequencial, seguindo as regras usuais de linguagens imperativas. Por exemplo, em uma atribuição, a expressão do lado direito é avaliada antes de seu valor ser associado à variável do lado esquerdo:

\begin{lstlisting}[numbers=none, language=C]
let x : int = 10;
x = x + 1;
\end{lstlisting}

Nesse caso, a expressão x + 1 é avaliada no estado atual do programa, resultando no valor 11, que é então atribuído à variável x, atualizando o ambiente de execução.

Expressões aritméticas, relacionais e booleanas são avaliadas de forma determinística, respeitando a precedência e associatividade definidas na gramática da linguagem. Considere o exemplo:

\begin{lstlisting}[numbers=none, language=C]
let result : bool = (3 + 2 * 4) > 10 && true;
\end{lstlisting}

A avaliação ocorre em etapas, primeiro resolvendo a multiplicação, seguida da soma, da comparação relacional e, por fim, da operação booleana, resultando no valor "true".

Estruturas de controle alteram o fluxo de execução com base na avaliação de expressões booleanas. No comando condicional if-else, apenas o bloco associado à condição verdadeira é executado:

\begin{lstlisting}[numbers=none, language=C]
if (x > 0) {
    print("positivo");
} else {
    print("nao positivo");
}

\end{lstlisting}

A escolha do bloco a ser executado depende exclusivamente do valor da expressão x > 0 no momento da avaliação.

Laços de repetição, como while, são executados enquanto a condição associada for verdadeira, produzindo múltiplas transições de estado:

\begin{lstlisting}[numbers=none, language=C]
let i : int = 0;
while (i < 3) {
    print(i);
    i = i + 1;
}
\end{lstlisting}

Nesse exemplo, o corpo do laço é executado três vezes, com a variável i sendo atualizada a cada iteração até que a condição i < 3 se torne falsa.

A chamada de funções em SL envolve a criação de um novo ambiente local, no qual os parâmetros formais são associados aos valores dos argumentos passados. Considere a função:

\begin{lstlisting}[numbers=none, language=C]
func inc(x : int) : int {
    return x + 1;
}
\end{lstlisting}

Ao executar a chamada inc(5), o valor 5 é associado ao parâmetro x em um novo ambiente, a expressão x + 1 é avaliada nesse contexto, e o valor 6 é retornado ao ponto de chamada, onde o ambiente local da função é então descartado.


\clearpage\newpage

\section{Arquitetura do Compilador}

\subsection{Gramatica do SL}
A gramática da linguagem SL define um programa como um conjunto de declarações globais, que podem corresponder a definições de funções ou de estruturas (structs). Essa organização reflete a natureza modular da linguagem, na qual funções e tipos compostos são definidos em nível global e utilizados ao longo do programa.

Para facilitar a explicação a gramática foi dividida em subconjuntos de regras referentes ao objetivo em comum que aquele conjunto de regras possui.

\subsubsection{Regras básicas:}
Definem a estrutura global de um programa em SL. Um programa é composto por uma lista de declarações globais, representadas por DeclList. Essas declarações podem aparecer em sequência e são restritas a definições de estruturas e funções, não sendo permitidos comandos ou expressões isoladas no nível global.

\begin{lstlisting}[mathescape=true, language=C]
Main -> DeclList
DeclList -> Decl DeclList | $\lambda$
Decl -> StructDef | FuncDef
\end{lstlisting}

\subsubsection{Definição de estruturas:} 
Uma estrutura é identificada por um nome e composta por um conjunto de campos, cada um associado a um tipo. A lista de campos pode conter zero ou mais declarações, permitindo estruturas vazias, e cada campo é definido por um identificador seguido de sua especificação de tipo.

\begin{lstlisting}[mathescape=true, language=C]
StructDef -> "struct" *ID* "{" FieldList "}"
FieldList -> FieldDecl FieldList | $\lambda$
FieldDecl -> *ID* ":" Type ";"
\end{lstlisting}

\subsubsection{Definição de função:}
Esse conjunto de regras define a sintaxe de definição de funções na linguagem SL, incluindo suporte a parâmetros genéricos e inferência parcial de tipos. Uma função pode opcionalmente declarar variáveis de tipo por meio de uma cláusula "forall", permitindo a definição de funções polimórficas. Cada função é identificada por um nome, recebe uma lista possivelmente vazia de parâmetros — que podem ou não ter seus tipos explicitamente declarados — e pode especificar opcionalmente um tipo de retorno.


\begin{lstlisting}[mathescape=true, language=C]
FuncDef -> GenericsDecl "func" *ID* "(" ParamList ")" OptReturnType "{" Block "}"
GenericsDecl -> "forall" IDlist "." | $\lambda$
IDlist -> *ID* IDlist | *ID*
ParamList -> Param "," ParamList | Param | $\lambda$
Param -> *ID* ":" Type | *ID*
OptReturnType -> ":" Type | $\lambda$
\end{lstlisting}

\subsubsection{Definição de tipos:}
Esse conjunto de regras define o sistema sintático de tipos da linguagem SL. A gramática contempla tipos básicos, como inteiros, booleanos, strings, etc. Como também listas com ou sem tamanho explicitamente especificado, e tipos de função, permitindo a descrição de assinaturas funcionais com múltiplos parâmetros e tipo de retorno.

\begin{lstlisting}[mathescape=true, language=C]
Type -> BaseType | Type "[" "]" | Type "[" Expr "]" | "(" TypeList ")" "->" Type
BaseType -> "int" | "float" | "string" | "bool" | "void" | *ID* |
TypeList -> Type "," TypeList | Type | $\lambda$
\end{lstlisting}
\subsubsection{Definição de comandos:}
Esse conjunto de regras define a sintaxe de comandos dentro de blocos de código na linguagem SL. Um bloco de código (Block) é composto por uma lista de instruções (StmtList), que podem ser variadas, incluindo declarações de variáveis (VarDecl), comandos de controle (if, while, for, e as estruturas "alternativas" como elif e else), chamadas de função (ReturnStmt), ou expressões simples (PrintStmt). 


\begin{lstlisting}[mathescape=true, language=C]
Block -> StmtList
StmtList -> Stmt StmtList | $\lambda$
Stmt -> VarDecl ";" | ReturnStmt ";" | PrintStmt ";" | Expr ";" | IfStmt | WhileStmt | ForStmt | Expr ";"
VarDecl -> "let" *ID* ":" Type OptInit
OptInit -> "=" Expr | $\lambda$ 
ReturnStmt -> "return" Expr | "return" 
PrintStmt -> "print" "(" Expr ")" 
IfStmt -> "if" "(" Expr ")" "{" Block "}" OptElif 
OptElif -> "elif" "(" Expr ")" "{" Block "}" OptElif | OptElse 
OptElse -> "else" "{" Block "}" | $\lambda$ 
WhileStmt -> "while" "(" Expr ")" "{" Block "}" 
ForStmt -> "for" "(" Expr ";" Expr ";" Expr ")" "{" Block "}"
\end{lstlisting}

\subsubsection{Definições de expressões:}
Esse conjunto de regras define a gramática de expressões da linguagem SL, organizando-as de forma hierárquica para garantir uma interpretação correta e não ambígua. A estrutura da gramática estabelece diferentes níveis de precedência, desde expressões de atribuição até expressões primárias, permitindo a combinação de operações aritméticas, relacionais e booleanas.

A gramática suporta atribuições, operações lógicas (||, \&\&), comparações, operações aritméticas, operadores unários, além de expressões mais complexas como acesso a elementos de arranjos, acesso a campos de estruturas, chamadas de função, criação de objetos e criação dinâmica de arranjos. O conceito de LValue é utilizado para restringir o lado esquerdo de uma atribuição a expressões válidas, como variáveis, acessos a arranjos ou campos de estruturas.

A definição recursiva das regras assegura a precedência e associatividade adequadas entre os operadores, evitando ambiguidades sintáticas. Além disso, a gramática permite expressões compostas e aninhadas, fornecendo uma base para a construção de programas complexos.



\begin{lstlisting}[mathescape=true, language=C]
Expr -> AssignExpr 
AssignExpr -> LValue "=" OrExpr | OrExpr 
OrExpr -> OrExpr "||" AndExpr | AndExpr 
AndExpr -> AndExpr "&&" CompExpr | CompExpr 
CompExpr ->  AddExpr CompOp AddExpr | AddExpr 
AddExpr -> AddExpr AddOp MultiExpr | MultiExpr 
MultiExpr -> MultiExpr MultiOp UnaryExpr | UnaryExpr 
UnaryExpr -> "!" UnaryExpr | "-" UnaryExpr | PrimaryExpr 

PrimaryExpr -> Literal | "[" ExprList "]" | LValue | FuncCall | ObjCreation | ArrayCreation | "(" Expr ")" | IncrementExpr | DecrementExpr
LValue -> *ID* | LValue "[" Expr "]" | LValue "." *ID* 
FuncCall -> *ID* "(" ExprList ")" 
ExprList -> Expr "," ExprList | Expr | $\lambda$ 
ObjCreation -> *ID* "{" ExprList "}" 
ArrayCreation -> *new* Type DimList
DimList -> "[" Expr "]" DimList | "[" Expr "]"
IncrementExpr -> LValue "++" 
DecrementExpr -> LValue "--"

\end{lstlisting}
\subsubsection{Terminais auxiliares:}
São os operadores utilizados nas expressões.

\begin{lstlisting}[mathescape=true, language=C]
CompOp -> "==" | "!=" | "<" | "<=" | ">" | ">=" 
AddOp -> "+" | "-" 
MultiOp -> "*" | "/" 
Literal -> *IntLit* | *FloatLit* | *StringLit* | *BoolLit*
\end{lstlisting}

\subsection{Análise léxica}
A análise léxica do compilador da linguagem SL é responsável por transformar o código-fonte em uma sequência de tokens, que representam as unidades léxicas fundamentais da linguagem. Essa etapa abstrai o fluxo de caracteres de entrada e fornece ao analisador sintático uma representação estruturada, contendo informações sobre o tipo do token e sua posição no código-fonte.

O analisador léxico foi implementado utilizando a ferramenta Alex, por meio do arquivo SL.x, que define os padrões léxicos da linguagem com base em expressões regulares. Cada padrão reconhecido é convertido em um token correspondente, conforme definido no módulo Token.hs.

\subsubsection{Estrutura de Tokens:}

Cada token é representado por um registro associado a uma lexema à sua posição no código fonte, permitindo  a emissão de mensagens de erro nas fases posteriores do compilador:

\begin{lstlisting}[language=haskell]
data Token = Token {
    pos :: (Int, Int),
    lexeme :: Lexeme
} deriving (Eq, Ord, Show)
\end{lstlisting}

O tipo Lexeme define todas as categorias léxicas da linguagem SL, incluindo literais, identificadores, delimitadores, operadores e palavras-chave.

\subsubsection{Literais e Identificadores:}

A gramática léxica contempla literais inteiros, de ponto flutuante, booleanos e strings, além de identificadores genéricos:


\begin{lstlisting}[language=haskell]
= TkIntLit Int
| TkFloatLit Float
| TkStringLit String
| TkBoolLit Bool
| TkID { out :: String }
[...]

\end{lstlisting}

\subsubsection{Operadores, Delimitadores e Palavras-chave:}

O analisador léxico reconhece operadores aritméticos, relacionais e booleanos, bem como símbolos de pontuação e delimitadores na sintaxe da linguagem:

\begin{lstlisting}[language=haskell]
[...]
    | TkLParen                  -- "("
    | TkRParen                  -- ")"
    | TkLBracket                -- "["
    | TkRBracket                -- "]"
    | TkLBrace                  -- "{"
    | TkRBrace                  -- "}"
[...]
    | TkAdd                     -- "+"
    | TkSub                     -- "-"
    | TkMul                     -- "*"
    | TkDiv                     -- "/"
[...]
\end{lstlisting}

Além disso, palavras-chave como func, struct, if, while, for, return e new são mapeadas para lexemas próprios, evitando ambiguidades na análise sintática.

\begin{lstlisting}[language=haskell]
[...]
    | TkIF
    | TkElif
    | TkElse
    | TkWhile
    | TkFor
[...]
\end{lstlisting}

\subsection{Análise Sintática}
A análise sintática é a fase responsável por verificar se a sequência de tokens produzida pelo analisador léxico obedece às regras gramaticais da linguagem SL. O resultado desse processo é a construção da Árvore de Sintaxe Abstrata (AST), uma estrutura de dados hierárquica que representa o programa livre de detalhes irrelevantes e pronta para as etapas de análise semântica e geração de código.

A implementação utiliza a ferramenta \textit{Happy} para gerar o parser a partir de uma gramática livre de contexto, e define a estrutura da AST em Haskell utilizando Tipos de Dados Algébricos (ADTs).

\subsection{Definição da Árvore de Sintaxe Abstrata (AST)}

A estrutura da AST foi definida no módulo \texttt{Frontend.Parser.Syntax}. O nó raiz da árvore é o tipo \texttt{SL}, que encapsula uma lista de declarações globais.

\begin{lstlisting}[language=haskell]
data SL = SL [Decl] deriving (Eq, Ord, Show)

data Decl
    = Struct ID [Field]
    | Func Generics ID [Param] (Maybe Type) Block
    deriving (Eq, Ord, Show)
\end{lstlisting}

Essa estrutura permite que um programa na linguagem SL seja composto por múltiplas definições de estruturas (\texttt{struct}) e funções (\texttt{func}), suportando inclusive tipos genéricos e parâmetros opcionais de retorno.

\subsection{Representação de Tipos}

O sistema de tipos da linguagem é representado pelo ADT \texttt{Type}. Além dos tipos primitivos (\texttt{int}, \texttt{float}, \texttt{bool}, etc.), a AST suporta tipos complexos recursivos, como arrays e funções (funções de primeira classe).

Note que a definição de \texttt{TyArray} permite armazenar uma expressão opcional para o tamanho, o que é crucial para validações semânticas futuras sobre o tamanho do vetor.

\begin{lstlisting}[language=haskell]
data Type 
    = TyInt | TyFloat | TyString | TyBool | TyVoid 
    | TyID ID 
    | TyArray Type (Maybe Expr)
    | TyFunc [Type] Type
    deriving (Eq, Ord, Show)
\end{lstlisting}

\subsection{Comandos e Fluxo de Controle}

Os comandos são representados pelo tipo \texttt{Stmt}. A estrutura reflete fielmente a gramática, onde blocos de controle como \texttt{if}, \texttt{while} e \texttt{for} encapsulam expressões de condição e blocos de execução.

Destaque para a estrutura do \texttt{IF}, que foi modelada para suportar uma lista de cláusulas \texttt{elif} e um bloco \texttt{else} opcional, facilitando a geração de código linearizada posteriormente.

\begin{lstlisting}[language=haskell]
data Stmt 
    = VarDecl ID (Maybe Type) (Maybe Expr) 
    | Return (Maybe Expr)
    | Print Expr
    | IF Expr Block [(Expr, Block)] (Maybe Block)
    | While Expr Block
    | For Expr Expr Expr Block
    | Exp Expr
    deriving (Eq, Ord, Show)
\end{lstlisting}

\subsection{Expressões e Precedência}

As expressões (\texttt{Expr}) formam o núcleo da lógica computacional. Elas são definidas recursivamente para suportar operações aritméticas, lógicas, relacionais e acessos a memória.

Para garantir a precedência correta dos operadores (por exemplo, multiplicação antes de adição), a AST utiliza construtores infixos e o arquivo de gramática (\texttt{SL.y}) define as prioridades explicitamente através das diretivas \texttt{\%left}, \texttt{\%right} e \texttt{\%nonassoc}.

\begin{lstlisting}[language=haskell]
data Expr
    = Expr := Expr      
    | Expr :||: Expr    
    | Expr :&&: Expr    
    | Expr :+: Expr    
    | Expr :*: Expr     
    
    | Expr :.: ID       
    | Expr :@: Expr     
    | FuncCall ID [Expr]
    | NewArray Type [Expr]
    -- ... (outros construtores)
    deriving (Eq, Ord, Show)

-- Ordem de Precedencia
infixr 1 :=
infixl 2 :||:
infixl 3 :&&:
infix  4 :==:, :!=:, :<:, :<=:, :>:, :>=:
infixl 6 :+:, :-:
infixl 7 :*:, :/:
infixl 8 :.:, :@:
\end{lstlisting}

\subsection{Especificação da Gramática (Happy)}

O analisador sintático foi gerado a partir do arquivo \texttt{Frontend.Parser.SL}. Ele foi configurado para operar dentro da Mônada \texttt{Alex}, permitindo uma comunicação direta e eficiente com o analisador léxico para controle de estado e posição (linha/coluna) em caso de erros.

A gramática resolve ambiguidades clássicas, como o problema do \textit{dangling else} e conflitos em declarações de array, através de regras de precedência. Um exemplo é a regra de criação de arrays, que utiliza \texttt{DimList} e precedência contextual (\texttt{\%prec LOW}) para diferenciar corretamente entre a criação de matrizes e acessos subsequentes.

\begin{lstlisting}[language=haskell]
ArrayCreation :: { Expr }
    : new BaseType DimList  { NewArray $2 $3 }

DimList :: { [Expr] }
    : '[' Expr ']' DimList         { $2 : $4 }
    | '[' Expr ']' %prec LOW       { [$2] }
\end{lstlisting}

A função de tratamento de erro (\texttt{parseError}) utiliza as informações de posição fornecidas pelo lexer para indicar exatamente onde a análise falhou, melhorando a experiência de depuração para o programador da linguagem SL.

\section{Resultados e Discussão}

\subsection{Instruções de Uso}


Para executar o protótipo realizado, basta, em um ambiente linux com as dependências necessárias, executar a seguinte linha de comando:

\begin{lstlisting}[mathescape=true, language=C]
    cabal sl -- [OPTIONS] [FILE]
\end{lstlisting}

Sendo "[FILE]" o caminho do arquivo a ser processado, e "[OPTIONS]" parâmetros opcionais para a compilação, sendo eles:

\begin{lstlisting}[mathescape=true, language=C]
      -l,  --lexer  <file>
      -pt, --parser <file>  
      -pp, --pretty <file>    
      -h, --help
\end{lstlisting}

\begin{itemize}
    \item (--lexer) Executa o analisador léxico no arquivo especificado;
    \item (--parser) Executa o analisador léxico e o parser no arquivo especificado, e imprime a ÁST; 
    \item (--pretty) Executa o analisador léxico e o parser no arquivo especificado, e utiliza a biblioteca pretty para imprime uma versão textual do programa;
    \item (-- help) Imprime uma mensagem explicando como executar o programa;
\end{itemize}


\subsection{Testes Realizados}
A garantia de qualidade do compilador SL foi conduzida em duas etapas complementares. A primeira fase consistiu em uma validação manual e exploratória, fundamental para estabilizar a gramática inicial. Nesta etapa, utilizou-se os algoritmos de exemplo fornecidos na especificação do trabalho como entrada para o compilador.

O processo de verificação envolveu a execução do analisador léxico e sintático e a inspeção visual de suas saídas. A lista de tokens gerada e, principalmente, a Árvore de Sintaxe Abstrata (AST) impressa no terminal foram meticulosamente comparadas com o código-fonte de entrada. Essa conferência manual permitiu assegurar que a estrutura hierárquica montada pelo parser correspondia fielmente à lógica dos programas originais, identificando erros de precedência e associações incorretas antes do desenvolvimento da suíte automatizada.

Após a validação inicial, foi desenvolvida uma suíte de testes unitários automatizados utilizando a biblioteca \texttt{Test.HUnit} do Haskell. Esta abordagem permitiu o refinamento da implementação, isolando componentes específicos do compilador para validar regras de precedência, tratamento de erros e construções complexas da linguagem.

\subsubsection{Testes do Analisador Léxico}

Para a validação do Lexer, foi criado um módulo específico (\texttt{Lexer.hs}) focado em garantir que todos os tokens fossem reconhecidos corretamente antes de serem enviados ao parser. A função auxiliar \texttt{scanMany} foi implementada para consumir toda a entrada e retornar a lista de tokens ou um erro léxico.

Os testes cobriram exaustivamente:
\begin{itemize}
    \item \textbf{Palavras-chave:} Verificação de todas as palavras reservadas, como \texttt{struct}, \texttt{forall}, \texttt{func}, entre outras.
    \item \textbf{Literais:} Reconhecimento correto de inteiros, pontos flutuantes, strings, booleanos e identificadores.
    \item \textbf{Operadores e Pontuação:} Testes de símbolos compostos (ex: \texttt{++}, \texttt{==}, \texttt{->}) para garantir que não houvesse ambiguidade com símbolos simples.
\end{itemize}

O código abaixo ilustra o teste de reconhecimento de palavras-chave:

\begin{lstlisting}[language=Haskell]
testKeywords :: Test
testKeywords = TestCase $ do
    let input = "struct forall func let return print if ..."
    let expected = [TkStruct, TkForAll, TkFunc, TkLet, TkReturn, ...]
    case scanMany input of
        Right tokens -> assertEqual "All keywords must be recognized" 
                        expected $ map tokenType tokens
        Left e -> assertFailure e
\end{lstlisting}

\subsubsection{Testes do Analisador Sintático}

A validação sintática, implementada no módulo \texttt{Parser.hs}, foi a etapa mais extensa. Para facilitar a depuração, foram criadas funções auxiliares como \texttt{parseStmt} e \texttt{parseExpr}, que encapsulam trechos de código dentro de uma função \texttt{main} fictícia, permitindo testar expressões e comandos isoladamente sem a necessidade de escrever um programa completo.

Foram elaborados testes específicos para garantir que a árvore sintática (AST) respeitasse a ordem correta das operações matemáticas e lógicas, como a precedência da multiplicação sobre a adição e do \texttt{AND} sobre o \texttt{OR}.

\begin{lstlisting}[language=Haskell]
testPrecedence :: Test
testPrecedence = TestCase $ do
    -- Teste: 2 + 3 * a.b
    -- Esperado: 2 + (3 * (a.b))
    let codeMath = "2 + 3 * a.b"
    let expMath  = LitInt 2 :+: (LitInt 3 :*: (Var "a" :.: "b"))

    case parseExpr codeMath of
        Right r1 -> assertEqual "Math Precedence (* over +)" expMath r1
        err -> assertFailure $ "Error: " ++ show err
\end{lstlisting}

Um foco importante foi a validação de estruturas de dados compostas. O parser foi submetido a testes de acesso a campos de \textit{structs} aninhadas (ex: \texttt{person.address.city}) e, crucialmente, ao uso de matrizes multidimensionais (ex: \texttt{matrix[i][j]}), assegurando que a recursividade da gramática para \texttt{LValue} e \texttt{ArrayCreation} funcionasse conforme o esperado.

\begin{lstlisting}[language=Haskell]
testComplexAccess :: Test
testComplexAccess = TestCase $ do
    -- Teste: matrix[i][j]
    let codeArray  = "matrix[i][j]"
    -- AST: (matrix[i])[j]
    let expArray   = (Var "matrix" :@: Var "i") :@: Var "j"

    -- Teste: users[0].name
    let codeMixed  = "users[0].name"
    let expMixed   = (Var "users" :@: LitInt 0) :.: "name"
\end{lstlisting}

Essa abordagem híbrida, combinando a execução dos algoritmos de exemplo com testes unitários granulares, garantiu robustez ao analisador, cobrindo desde a detecção básica de tokens até a construção correta da AST em casos de alta complexidade sintática.

\subsection{Limitações}
Devido à complexidade inerente ao desenvolvimento de um compilador e às restrições de tempo, foram necessárias simplificações estratégicas no design da linguagem e na implementação do analisador. No que tange à semântica, o sistema de tipos não contempla a generalização ou inferência para estruturas (\textit{structs}), exigindo tipagem explícita. Além disso, optou-se por implementar matrizes multidimensionais como "vetores de vetores" (\textit{jagged arrays}) em vez de blocos contíguos de memória, e o comando \texttt{print} foi definido como uma instrução sintática estrita. Embora funcional, essa decisão impede que o \texttt{print} seja tratado como um "cidadão de primeira classe" como as outras funções, impossibilitando seu uso como parâmetro em funções de alta ordem ou sua atribuição a variáveis, diferentemente do que ocorre em linguagens puramente funcionais como Haskell.

No aspecto da ferramenta e interface de depuração, a visualização da Árvore de Sintaxe Abstrata (acionada pela opção \texttt{--parser}) apresenta limitações na codificação de caracteres. O mecanismo de impressão atual não suporta nativamente a saída em UTF-8, o que pode resultar na exibição incorreta de caracteres acentuados ou especiais no terminal durante a inspeção da árvore, embora o reconhecimento desses caracteres pelo analisador léxico ocorra corretamente internamente.

\section{Conclusão}
O desenvolvimento das etapas de análise léxica e sintática do compilador para a linguagem SL atingiu os objetivos propostos, resultando na construção de um \textit{front-end} funcional e robusto. Através desta implementação, foi possível consolidar na prática os conceitos teóricos fundamentais da disciplina, traduzindo especificações de linguagens formais e autômatos em uma ferramenta concreta capaz de reconhecer e estruturar programas complexos.

O uso de Tipos de Dados Algébricos (ADTs) permitiu uma modelagem concisa da Árvore de Sintaxe Abstrata (AST), capturando a recursividade inerente às expressões e declarações da linguagem SL. Adicionalmente, a abordagem monádica adotada para a integração entre o lexer e o parser foi importante para o gerenciamento de estado e para um sistema de tratamento de erros eficiente, capaz de reportar falhas retornando linha e coluna.

Os desafios técnicos encontrados, como a resolução de conflitos de precedência em expressões, o tratamento de ambiguidades clássicas (como o \textit{dangling else}) e as decisões de design referentes à representação de arrays multidimensionais, foram superados através de um refinamento iterativo da gramática. A metodologia de testes, evoluindo de uma verificação manual exploratória para um conjunto de testes unitários automatizados, garantiu a estabilidade do analisador diante de diversos cenários de entrada.

Embora o projeto apresente limitações pontuais, como a ausência de inferência para estruturas e restrições na visualização de caracteres Unicode na árvore sintática, a estrutura entregue cumpre rigorosamente os requisitos da etapa. A AST produzida possui informações semânticas e estruturais suficientes para estabelecer uma base sólida para as fases subsequentes de Análise Semântica e Geração de Código para WebAssembly (WAT), viabilizando a continuidade do desenvolvimento da linguagem SL.

\section{Referências}

\begin{thebibliography}{9}
\bibitem{hutton2016}
Hutton, G. (2016). \emph{Programming in Haskell}. Cambridge University Press.

\bibitem{appel1998}
Appel, A. W. (1998). \emph{Modern Compiler Implementation in ML}. Cambridge University Press.

\bibitem{webassembly2023}
WebAssembly Community Group. (2023). \emph{WebAssembly Specification}.
\end{thebibliography}

\end{document}
