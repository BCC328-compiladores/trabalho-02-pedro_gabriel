O estudo e o desenvolvimento de compiladores é um dos temas mais importantes da Ciência da Computação, pois envolve a integração de conceitos fundamentais da teoria da computação, como linguagens formais, teoria de autômatos, estrutura de dados, algoritmos e arquitetura de computadores. Um compilador é responsável por traduzir programas escritos em uma linguagem de alto nível para uma representação de mais baixo nível, preservando a semântica do programa original e permitindo sua execução.

Este trabalho tem como objetivo o desenvolvimento de um compilador para a linguagem SL, uma linguagem simplificada, com suporte a tipagem estática, funções, estruturas de controle, arranjos e registros. O compilador será implementado na linguagem Haskell, e terá como linguagem-alvo o WebAssembly (WAT), possibilitando a execução dos programas gerados em navegadores web.

O projeto foi sub-dividido em etapas incrementais: Na primeira etapa, foco deste relatório, são abordadas as fases de análise léxica e análise sintática, responsáveis por transformar o código-fonte em uma estrutura sintática que represente formalmente o programa de entrada. Para isso, foram utilizadas as ferramentas Alex e Happy, amplamente empregadas na costrução de compiladores em Haskell.

Ao longo deste relatório, são descritas as principais decisões de projeto adotadas, a gramática definida para a linguagem SL e a arquitetura geral dos analisadores léxico e sintático. O objetivo é não apenas apresentar a implementação, mas também justificar as escolhas realizadas, relacionando-as com os conceitos teóricos estudados na disciplina de Construção de Compiladores.