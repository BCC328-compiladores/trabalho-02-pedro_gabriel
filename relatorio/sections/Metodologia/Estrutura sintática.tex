Definir a estrutura sintática da linguagem é um passo fundamental no desenvolvimento do compilador, pois estabelece formalmente como programas válidos da linguagem devem ser escritos.

A linguagem SL adota uma sintaxe com suporte a declarações explícitas de variáveis, definição de funções, estruturas de controle (como condicionais e laços de repetição), além de arranjos e registros. Cada programa é composto por um conjunto de definições globais, que podem ser declarações de funções ou de estruturas. As funções possuem parâmetros, tipo de retorno e um corpo delimitado por blocos.
