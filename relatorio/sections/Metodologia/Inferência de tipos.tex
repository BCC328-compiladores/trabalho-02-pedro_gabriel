A linguagem SL foi projetada com suporte à inferência de tipos, permitindo que o compilador deduza automaticamente o tipo de expressões e identificadores a partir do contexto em que são utilizados. Esse mecanismo reduz a necessidade de anotações explícitas por parte do programador, tornando o código mais conciso, sem comprometer a segurança.

A inferência de tipos em SL baseia-se na análise das expressões durante a construção e posterior anotação da Árvore de Sintaxe Abstrata (AST). A cada nó da AST é associado um tipo, que pode ser inicialmente desconhecido e progressivamente refinado à medida que regras de tipagem são aplicadas. Por exemplo, em uma expressão aritmética, o tipo do resultado é inferido a partir dos tipos de seus operandos e do operador utilizado.

No caso de variáveis locais, o tipo pode ser inferido diretamente a partir da expressão de inicialização. Já em chamadas de função, o compilador verifica a compatibilidade entre os tipos dos argumentos e os tipos esperados pelos parâmetros da função, propagando o tipo de retorno conforme definido ou inferido. Para funções genéricas, o processo envolve a instância de variáveis de tipo, respeitando as restrições impostas pelo uso da função.

Apesar do suporte à inferência, SL exige anotações explícitas em pontos estruturais relevantes, como definições de funções globais e declarações de estruturas. Como ilustração, considere o código a seguir:

\begin{lstlisting}[numbers=none, language=C]
let x = 10;
let y = x + 2;
\end{lstlisting}

Neste exemplo, o compilador infere o tipo mais apropriado para as variáveis x e y, que no caso é o tipo inteiro.