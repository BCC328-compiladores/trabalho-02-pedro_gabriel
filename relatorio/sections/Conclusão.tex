O desenvolvimento das etapas de análise léxica e sintática do compilador para a linguagem SL atingiu os objetivos propostos, resultando na construção de um \textit{front-end} funcional e robusto. Através desta implementação, foi possível consolidar na prática os conceitos teóricos fundamentais da disciplina, traduzindo especificações de linguagens formais e autômatos em uma ferramenta concreta capaz de reconhecer e estruturar programas complexos.

O uso de Tipos de Dados Algébricos (ADTs) permitiu uma modelagem concisa da Árvore de Sintaxe Abstrata (AST), capturando a recursividade inerente às expressões e declarações da linguagem SL. Adicionalmente, a abordagem monádica adotada para a integração entre o lexer e o parser foi importante para o gerenciamento de estado e para um sistema de tratamento de erros eficiente, capaz de reportar falhas retornando linha e coluna.

Os desafios técnicos encontrados, como a resolução de conflitos de precedência em expressões, o tratamento de ambiguidades clássicas (como o \textit{dangling else}) e as decisões de design referentes à representação de arrays multidimensionais, foram superados através de um refinamento iterativo da gramática. A metodologia de testes, evoluindo de uma verificação manual exploratória para um conjunto de testes unitários automatizados, garantiu a estabilidade do analisador diante de diversos cenários de entrada.

Embora o projeto apresente limitações pontuais, como a ausência de inferência para estruturas e restrições na visualização de caracteres Unicode na árvore sintática, a estrutura entregue cumpre rigorosamente os requisitos da etapa. A AST produzida possui informações semânticas e estruturais suficientes para estabelecer uma base sólida para as fases subsequentes de Análise Semântica e Geração de Código para WebAssembly (WAT), viabilizando a continuidade do desenvolvimento da linguagem SL.