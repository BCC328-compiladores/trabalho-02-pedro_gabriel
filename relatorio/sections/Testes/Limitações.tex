Devido à complexidade inerente ao desenvolvimento de um compilador e às restrições de tempo, foram necessárias simplificações estratégicas no design da linguagem e na implementação do analisador. No que tange à semântica, o sistema de tipos não contempla a generalização ou inferência para estruturas (\textit{structs}), exigindo tipagem explícita. Além disso, optou-se por implementar matrizes multidimensionais como "vetores de vetores" (\textit{jagged arrays}) em vez de blocos contíguos de memória, e o comando \texttt{print} foi definido como uma instrução sintática estrita. Embora funcional, essa decisão impede que o \texttt{print} seja tratado como um "cidadão de primeira classe" como as outras funções, impossibilitando seu uso como parâmetro em funções de alta ordem ou sua atribuição a variáveis, diferentemente do que ocorre em linguagens puramente funcionais como Haskell.

No aspecto da ferramenta e interface de depuração, a visualização da Árvore de Sintaxe Abstrata (acionada pela opção \texttt{--parser}) apresenta limitações na codificação de caracteres. O mecanismo de impressão atual não suporta nativamente a saída em UTF-8, o que pode resultar na exibição incorreta de caracteres acentuados ou especiais no terminal durante a inspeção da árvore, embora o reconhecimento desses caracteres pelo analisador léxico ocorra corretamente internamente.